
\documentclass[10pt,a4paper]{report}
\usepackage[utf8]{inputenc}
\usepackage[russian]{babel}
\usepackage{amsmath}
\usepackage{amsfonts}
\usepackage{amssymb}
\usepackage{graphicx}
\author{Скрипаль Борис}
\title{Лабораторная работа №2.\\
	Утилита nmap}
\begin{document}
	\maketitle
	\renewcommand{\thesection}{\arabic{section}}
	\tableofcontents
	\pagebreak
	
	\setcounter{totalnumber}{10}
	\setcounter{topnumber}{10}
	\setcounter{bottomnumber}{10}
	\renewcommand{\topfraction}{1}
	\renewcommand{\textfraction}{0}
	
	\section{Описание лабораторных условий}
		Для проведения лабораторной работы было создано две виртуальные машины, объединенные в общую виртуальную сеть. Первая виртуальная машина (Metasploitable2) намеренно содержит ряд уязвимостей. Вторая виртуальная машина (Kali Linux) необходима для сканирования и поска уязвимостей на первой виртуальной машине.
		
		Виртуальная машина с ОС Metasploitable2 имеет адрес 192.168.202.2, вторая (с Kali Linux) имеет адрес 192.168.202.3.
		
		Конфигурация первой машины:
		\begin{verbatim}
			msfadmin@metasploitable:~$ ifconfig -a
			eth0      Link encap:Ethernet  HWaddr 08:00:27:3b:18:a4
			inet addr:192.168.202.2  Bcast:0.0.0.0  Mask:255.255.255.255
			inet6 addr: fe80::a00:27ff:fe3b:18a4/64 Scope:Link
			UP BROADCAST RUNNING MULTICAST  MTU:1500  Metric:1
			RX packets:269 errors:0 dropped:0 overruns:0 frame:0
			TX packets:142 errors:0 dropped:0 overruns:0 carrier:0
			collisions:0 txqueuelen:1000
			RX bytes:24841 (24.2 KB)  TX bytes:22808 (22.2 KB)
			Base address:0xd010 Memory:f0000000-f0020000
		\end{verbatim}
		Конфигурация второй машины:
		\begin{verbatim}
			root@kali:~# ifconfig -a
			eth0: flags=4163<UP,BROADCAST,RUNNING,MULTICAST>  mtu 1500
			inet 192.168.202.3  netmask 255.255.255.0  broadcast 192.168.202.255
			inet6 fe80::a00:27ff:fe35:a17a  prefixlen 64  scopeid 0x20<link>
			ether 08:00:27:35:a1:7a  txqueuelen 1000  (Ethernet)
			RX packets 7  bytes 496 (496.0 B)
			RX errors 0  dropped 0  overruns 0  frame 0
			TX packets 51  bytes 6218 (6.0 KiB)
			TX errors 0  dropped 0 overruns 0  carrier 0  collisions 0
		\end{verbatim}
		Для проверки работоспособности сети "пропингуем" виртуальные машины:
		\begin{verbatim}
			msfadmin@metasploitable:~$ ping 192.168.202.3
			PING 192.168.202.3 (192.168.202.3) 56(84) bytes of data.
			64 bytes from 192.168.202.3: icmp_seq=1 ttl=64 time=10.5 ms
			64 bytes from 192.168.202.3: icmp_seq=2 ttl=64 time=0.609 ms
			
			--- 192.168.202.3 ping statistics ---
			2 packets transmitted, 2 received, 0% packet loss, time 1001ms
			rtt min/avg/max/mdev = 0.609/5.564/10.520/4.956 ms
		\end{verbatim}
		\begin{verbatim}
			root@kali:~# ping 192.168.202.2
			PING 192.168.202.2 (192.168.202.2) 56(84) bytes of data.
			64 bytes from 192.168.202.2: icmp_seq=1 ttl=64 time=0.325 ms
			64 bytes from 192.168.202.2: icmp_seq=2 ttl=64 time=0.514 ms
			^C
			--- 192.168.202.2 ping statistics ---
			2 packets transmitted, 2 received, 0% packet loss, time 999ms
			rtt min/avg/max/mdev = 0.325/0.419/0.514/0.096 ms
		\end{verbatim}
		Как видно из результатов, оба хоста находятся в одной сети и видят друг друга.
	\section{Поиск активных хостов}
		Для поиска активных хостов можно воспользоваться флагами -sP утилиты nmap. При задании данных флагов, утилита не сканирует порты, а ищет только активные хосты в сети. Ввиду того, что маска нашей подсети /24, то для того, что бы утилита просмотрела всю подсеть, необходимо задать ей адрес подсети. В нашем случае 192.168.202.
		
		Результат сканирования подсети на наличие активных хостов:
		\begin{verbatim}
			root@kali:~# nmap -sP 192.168.202.*
			Starting Nmap 7.01 ( https://nmap.org ) at 2016-03-20 16:39 EDT
			Nmap scan report for 192.168.202.1
			Host is up (0.00053s latency).
			MAC Address: 0A:00:27:00:00:2C (Unknown)
			Nmap scan report for 192.168.202.2
			Host is up (0.0013s latency).
			MAC Address: 08:00:27:3B:18:A4 (Oracle VirtualBox virtual NIC)
			Nmap scan report for 192.168.202.3
			Host is up.
			Nmap done: 256 IP addresses (3 hosts up) scanned in 28.13 seconds
		\end{verbatim}
		В результате сканирования, утилита нашла 3 активных хоста: адрес первого хоста - адрес адаптера виртуальной сети, второй - адрес виртуальной машины metasploitable, а третий адрес - адрес текущей машины.
	\section{Сканирование портов хоста}
		Для сканирования портов хоста, утилите nmap необходимо передать адрес хоста, например:
		\begin{verbatim}
			root@kali:~# nmap 192.168.202.2
			Starting Nmap 7.01 ( https://nmap.org ) at 2016-03-20 16:42 EDT
			Nmap scan report for 192.168.202.2
			Host is up (0.00022s latency).
			Not shown: 977 closed ports
			PORT     STATE SERVICE
			21/tcp   open  ftp
			22/tcp   open  ssh
			23/tcp   open  telnet
			25/tcp   open  smtp
			53/tcp   open  domain
			80/tcp   open  http
			111/tcp  open  rpcbind
			139/tcp  open  netbios-ssn
			445/tcp  open  microsoft-ds
			512/tcp  open  exec
			513/tcp  open  login
			514/tcp  open  shell
			1099/tcp open  rmiregistry
			1524/tcp open  ingreslock
			2049/tcp open  nfs
			2121/tcp open  ccproxy-ftp
			3306/tcp open  mysql
			5432/tcp open  postgresql
			5900/tcp open  vnc
			6000/tcp open  X11
			6667/tcp open  irc
			8009/tcp open  ajp13
			8180/tcp open  unknown
			MAC Address: 08:00:27:3B:18:A4 (Oracle VirtualBox virtual NIC)
			Nmap done: 1 IP address (1 host up) scanned in 13.35 seconds
		\end{verbatim}
		Как видно из вывода, на исследуемой виртуальной машине открыты 21 порт (ftp), 22 порт (ssh), 23 порт(telnet), 80 порт (http), а так же ряд других портов.
	\section{Исследование служебных файлов nmap-services, nmap-os-db, nmap-service-probes}
		Служебные файлы для утилиты nmap по умолчанию располагаются в директории "/usr/share/nmap".
		\subsection{Файл nmap-services}
			Служебный файл nmap-services содержит в себе описание назначения стандартных портов. Сам файл имеет структуру таблицы со следующими столбцами: имя\_сервиса, номер\_порта/название\_протокола, частота, комментарии.
			
			Для известных, зарезервированных номеров портов, файл содержит подробное описание, например:
			\begin{verbatim}
				ftp-data	20/sctp	0.000000	# File Transfer [Default Data]
				ftp-data	20/tcp	0.001079	# File Transfer [Default Data]
				ftp-data	20/udp	0.001878	# File Transfer [Default Data]
				ftp	21/sctp	0.000000	# File Transfer [Control]
				ftp	21/tcp	0.197667	# File Transfer [Control]
				ftp	21/udp	0.004844	# File Transfer [Control]
				ssh	22/sctp	0.000000	# Secure Shell Login
				ssh	22/tcp	0.182286	# Secure Shell Login
				ssh	22/udp	0.003905	# Secure Shell Login
			\end{verbatim}
			Для "свободных" номеров портов, файл так же содержит записи, но они не несут никакой полезной информации, что ожидаемо, так как на этих портах запускаются пользовательские сервисы, и они не закреплены ни за одним приложением.
			\begin{verbatim}
				unknown	26860/udp	0.000654
				unknown	26861/udp	0.000654
				unknown	26866/udp	0.001307
				unknown	26868/udp	0.001307
			\end{verbatim}
		\subsection{Файл nmap-os-db}
			Данный файл содержит сигнатуры ответов различных операционных систем при сканировании. Это необходимо для того, что бы узнать какая операционная система находится на данном хосте. Пример файла nmap-os-db:
			\begin{verbatim}
				MatchPoints
				SEQ(SP=25%GCD=75%ISR=25%TI=100%CI=50%II=100%SS=80%TS=100)
				OPS(O1=20%O2=20%O3=20%O4=20%O5=20%O6=20)
				WIN(W1=15%W2=15%W3=15%W4=15%W5=15%W6=15)
				ECN(R=100%DF=20%T=15%TG=15%W=15%O=15%CC=100%Q=20)
			\end{verbatim}
			Для того, что бы утилита nmap выводила операционную систему, необходимо задать ключ -O:
			\begin{verbatim}
				root@kali:~# nmap -O 192.168.202.2
				Starting Nmap 7.01 ( https://nmap.org ) at 2016-03-20 17:51 EDT
				Nmap scan report for 192.168.202.2
				Host is up (0.00064s latency).
				Not shown: 977 closed ports
				PORT     STATE SERVICE
				21/tcp   open  ftp
				22/tcp   open  ssh
				23/tcp   open  telnet
				25/tcp   open  smtp
				53/tcp   open  domain
				80/tcp   open  http
				111/tcp  open  rpcbind
				139/tcp  open  netbios-ssn
				445/tcp  open  microsoft-ds
				512/tcp  open  exec
				513/tcp  open  login
				514/tcp  open  shell
				1099/tcp open  rmiregistry
				1524/tcp open  ingreslock
				2049/tcp open  nfs
				2121/tcp open  ccproxy-ftp
				3306/tcp open  mysql
				5432/tcp open  postgresql
				5900/tcp open  vnc
				6000/tcp open  X11
				6667/tcp open  irc
				8009/tcp open  ajp13
				8180/tcp open  unknown
				MAC Address: 08:00:27:3B:18:A4 (Oracle VirtualBox virtual NIC)
				Device type: general purpose
				Running: Linux 2.6.X
				OS CPE: cpe:/o:linux:linux_kernel:2.6
				OS details: Linux 2.6.9 - 2.6.33
				Network Distance: 1 hop
			\end{verbatim}
			Как видно из вывода утилиты, на хосте располагается операционная система Linux с версией ядра 2.6.
		\subsection{Файл nmap-service-probes}
			Данный файл содержит сигнатуры для определения сервисов, прослушивающих тот или иной порт. Как правило, это относится к известным службам, например SMTP - почтовый серве, или DNS, или какой-либо другой широко используемый сервис. Данный о сервисах задаются при помощи нескольких директив:
			\begin{itemize}
				\item Exclude <port specification>
				\item Probe <protocol> <probename> <probestring>
				\item match <service> <pattern> [<versioninfo>]
			\end{itemize}
	\section{Добавление собственной сигнатуры в файл nmap-service-probes}
		Для того, что бы добавить собственную сигнатуру, создадим небольшой сервер, который мы будем идентифицировать при помощи утилиты nmap. Код сервера представлен ниже:
		\begin{verbatim}
			#include <sys/types.h>
			#include <sys/socket.h>
			#include <netdb.h>
			#include <stdio.h>
			#include<string.h>
			
			int main()
			{
			
			char str[100];
			char *rec_srt="Hello, client (MyServer 1.1)\n";
			
			int listen_fd, comm_fd;
			
			struct sockaddr_in servaddr;
			
			listen_fd = socket(AF_INET, SOCK_STREAM, 0);
			
			bzero( &servaddr, sizeof(servaddr));
			
			servaddr.sin_family = AF_INET;
			servaddr.sin_addr.s_addr = htons(INADDR_ANY);
			servaddr.sin_port = htons(22000);
			
			bind(listen_fd, (struct sockaddr *) &servaddr, sizeof(servaddr));
			
			listen(listen_fd, 10);
			
			comm_fd = accept(listen_fd, (struct sockaddr*) NULL, NULL);
			
			while(1)
			{
			
			bzero( str, 100);
			
			read(comm_fd,str,100);
			
			printf("Echoing back - %s",str);
			
			write(comm_fd, rec_srt, strlen(rec_srt)+1);
			
			}
			}
		\end{verbatim}
		Сервер слушает порт 22000 и на входящее сообщение отправляет приветственное сообщение со своей версией.
		
		Для определения данного сервера, добавим в файл nmap-service-probes следующие строки:
		\begin{verbatim}
			Probe TCP MyServer q|\x02Hi|
			rarity 1
			ports 22000
			match testServer m/^Hello, client \((\w*) ([\d.]*)\)/ p/$1/ v/$2/
		\end{verbatim}
		
		В данных строках мы описываем, какое сообщение будем отправлять для идентификации, на какой порт, а так же какой ответ мы будем ожидать.
		
		Запустим на испытуемой виртуальной машине данный сервер и попробуем определить его при помощи утилиты nmap:
		\begin{verbatim}
			root@kali:~/test# nmap 192.168.202.2 -p 22000 -sV
			
			Starting Nmap 7.01 ( https://nmap.org ) at 2016-03-20 19:33 EDT
			Nmap scan report for 192.168.202.2
			Host is up (0.00048s latency).
			PORT      STATE SERVICE    VERSION
			22000/tcp open  testServer MyServer 1.1
			MAC Address: 08:00:27:3B:18:A4 (Oracle VirtualBox virtual NIC)
			
			Service detection performed. Please report any incorrect results
			at https://nmap.org/submit/ .
			Nmap done: 1 IP address (1 host up) scanned in 20.05 seconds
		\end{verbatim}
		Как видно из вывода, утилита корректно определила наш сервер.
	\section{Сохранение вывода утилиты nmap в формате XML}
		Для сохранения вывода утилиты nmap в формате  XML необходимо при 
		запуске указать ключ "-oX", например:
		\begin{verbatim}
			nmap -T4 -A -p 1-5000 -oX - 192.168.202.2 > out.xml
		\end{verbatim}
		В результате запуска, утилитой nmap будут просканированы порты с 1 по 
		5000 хоста с адресом 192.168.202.2, и вывод будет сохранен в формате 
		XML. Содержимое файла out.xml:
		\begin{verbatim}
			<?xml version="1.0" encoding="UTF-8"?>
			<!DOCTYPE nmaprun>
			<?xml-stylesheet href="file:///usr/bin/../share/nmap/nmap.xsl" 
			type="text/xsl"?>
			<!-- Nmap 7.01 scan initiated Mon Mar 21 04:37:35 2016 as: nmap -T4 
			-A -p 1-5000 -oX - 192.168.202.2 -->
			<nmaprun scanner="nmap" args="nmap -T4 -A -p 1-5000 -oX - 
			192.168.202.2" start="1458549455" startstr="Mon Mar 21 04:37:35 
			2016" version="7.01" xmloutputversion="1.04">
			<scaninfo type="syn" protocol="tcp" numservices="5000" 
			services="1-5000"/>
			<verbose level="0"/>
			<debugging level="0"/>
			<host starttime="1458549455" endtime="1458549492"><status 
			state="up" reason="arp-response" reason_ttl="0"/>
			<address addr="192.168.202.2" addrtype="ipv4"/>
			<address addr="08:00:27:3B:18:A4" addrtype="mac" vendor="Oracle 
			VirtualBox virtual NIC"/>
			<hostnames>
			</hostnames>
			<ports><extraports state="closed" count="4982">
			<extrareasons reason="resets" count="4982"/>
			</extraports>
			<port protocol="tcp" portid="21"><state state="open" 
			reason="syn-ack" reason_ttl="64"/><service name="ftp" 
			product="vsftpd" version="2.3.4" ostype="Unix" method="probed" 
			conf="10"><cpe>cpe:/a:vsftpd:vsftpd:2.3.4</cpe></service><script 
			id="ftp-anon" output="Anonymous FTP login allowed (FTP code 
			230)"/></port>
			<port protocol="tcp" portid="22"><state state="open" 
			reason="syn-ack" reason_ttl="64"/><service name="ssh" 
			product="OpenSSH" version="4.7p1 Debian 8ubuntu1" 
			extrainfo="protocol 2.0" ostype="Linux" method="probed" 
			conf="10"><cpe>cpe:/a:openbsd:openssh:4.7p1</cpe><cpe>cpe:/o:linux:linux_kernel</cpe></service><script
			 id="ssh-hostkey" output="&#xa;  1024 
			60:0f:cf:e1:c0:5f:6a:74:d6:90:24:fa:c4:d5:6c:cd (DSA)&#xa;  2048 
			56:56:24:0f:21:1d:de:a7:2b:ae:61:b1:24:3d:e8:f3 (RSA)"><table>
			<elem key="fingerprint">600fcfe1c05f6a74d69024fac4d56ccd</elem>
			<elem 
			key="key">AAAAB3NzaC1kc3MAAACBALz4hsc8a2Srq4nlW960qV8xwBG0JC+jI7fWxm5METIJH4tKr/xUTwsTYEYnaZLzcOiy21D3ZvOwYb6AA3765zdgCd2Tgand7F0YD5UtXG7b7fbz99chReivL0SIWEG/E96Ai+pqYMP2WD5KaOJwSIXSUajnU5oWmY5x85sBw+XDAAAAFQDFkMpmdFQTF+oRqaoSNVU7Z+hjSwAAAIBCQxNKzi1TyP+QJIFa3M0oLqCVWI0We/ARtXrzpBOJ/dt0hTJXCeYisKqcdwdtyIn8OUCOyrIjqNuA2QW217oQ6wXpbFh+5AQm8Hl3b6C6o8lX3Ptw+Y4dp0lzfWHwZ/jzHwtuaDQaok7u1f971lEazeJLqfiWrAzoklqSWyDQJAAAAIA1lAD3xWYkeIeHv/R3P9i+XaoI7imFkMuYXCDTq843YU6Td+0mWpllCqAWUV/CQamGgQLtYy5S0ueoks01MoKdOMMhKVwqdr08nvCBdNKjIEd3gH6oBk/YRnjzxlEAYBsvCmM4a0jmhz0oNiRWlc/F+bkUeFKrBx/D2fdfZmhrGg==</elem>
			<elem key="type">ssh-dss</elem>
			<elem key="bits">1024</elem>
			</table>
			<table>
			<elem key="fingerprint">5656240f211ddea72bae61b1243de8f3</elem>
			<elem 
			key="key">AAAAB3NzaC1yc2EAAAABIwAAAQEAstqnuFMBOZvO3WTEjP4TUdjgWkIVNdTq6kboEDjteOfc65TlI7sRvQBwqAhQjeeyyIk8T55gMDkOD0akSlSXvLDcmcdYfxeIF0ZSuT+nkRhij7XSSA/Oc5QSk3sJ/SInfb78e3anbRHpmkJcVgETJ5WhKObUNf1AKZW++4Xlc63M4KI5cjvMMIPEVOyR3AKmI78Fo3HJjYucg87JjLeC66I7+dlEYX6zT8i1XYwa/L1vZ3qSJISGVu8kRPikMv/cNSvki4j+qDYyZ2E5497W87+Ed46/8P42LNGoOV8OcX/ro6pAcbEPUdUEfkJrqi2YXbhvwIJ0gFMb6wfe5cnQew==</elem>
			<elem key="type">ssh-rsa</elem>
			<elem key="bits">2048</elem>
			</table>
			</script></port>
			<port protocol="tcp" portid="23"><state state="open" 
			reason="syn-ack" reason_ttl="64"/><service name="telnet" 
			product="Linux telnetd" ostype="Linux" method="probed" 
			conf="10"><cpe>cpe:/o:linux:linux_kernel</cpe></service></port>
			<port protocol="tcp" portid="25"><state state="open" 
			reason="syn-ack" reason_ttl="64"/><service name="smtp" 
			product="Postfix smtpd" hostname=" metasploitable.localdomain" 
			method="probed" 
			conf="10"><cpe>cpe:/a:postfix:postfix</cpe></service><script 
			id="smtp-commands" output="metasploitable.localdomain, PIPELINING, 
			SIZE 10240000, VRFY, ETRN, STARTTLS, ENHANCEDSTATUSCODES, 8BITMIME, 
			DSN, "/><script id="ssl-cert" output="Subject: 
			commonName=ubuntu804-base.localdomain/organizationName=OCOSA/stateOrProvinceName=There
			 is no such thing outside US/countryName=XX&#xa;Not valid before: 
			2010-03-17T14:07:45&#xa;Not valid after:  
			2010-04-16T14:07:45"><table key="subject">
			<elem key="stateOrProvinceName">There is no such thing outside 
			US</elem>
			<elem key="organizationName">OCOSA</elem>
			<elem key="countryName">XX</elem>
			<elem key="localityName">Everywhere</elem>
			<elem key="organizationalUnitName">Office for Complication of 
			Otherwise Simple Affairs</elem>
			<elem key="commonName">ubuntu804-base.localdomain</elem>
			<elem key="emailAddress">root@ubuntu804-base.localdomain</elem>
			</table>
			<table key="issuer">
			<elem key="stateOrProvinceName">There is no such thing outside 
			US</elem>
			<elem key="organizationName">OCOSA</elem>
			<elem key="countryName">XX</elem>
			<elem key="localityName">Everywhere</elem>
			<elem key="organizationalUnitName">Office for Complication of 
			Otherwise Simple Affairs</elem>
			<elem key="commonName">ubuntu804-base.localdomain</elem>
			<elem key="emailAddress">root@ubuntu804-base.localdomain</elem>
			</table>
			<table key="pubkey">
			<elem key="type">rsa</elem>
			<elem key="bits">1024</elem>
			</table>
			<elem key="sig_algo">sha1WithRSAEncryption</elem>
			<table key="validity">
			<elem key="notBefore">2010-03-17T14:07:45</elem>
			<elem key="notAfter">2010-04-16T14:07:45</elem>
			</table>
			<elem key="md5">dcd9ad906c8f2f7374af383b25408828</elem>
			<elem key="sha1">ed093088706603bfd5dc237399b498da2d4d31c6</elem>
			<elem key="pem">-&#45;&#45;&#45;&#45;BEGIN 
			CERTIFICATE-&#45;&#45;&#45;&#45;&#xa;MIIDWzCCAsQCCQD6+TpMf7a5zDANBgkqhkiG9w0BAQUFADCB8TELMAkGA1UEBhMC&#xa;WFgxKjAoBgNVBAgTIVRoZXJlIGlzIG5vIHN1Y2ggdGhpbmcgb3V0c2lkZSBVUzET&#xa;MBEGA1UEBxMKRXZlcnl3aGVyZTEOMAwGA1UEChMFT0NPU0ExPDA6BgNVBAsTM09m&#xa;ZmljZSBmb3IgQ29tcGxpY2F0aW9uIG9mIE90aGVyd2lzZSBTaW1wbGUgQWZmYWly&#xa;czEjMCEGA1UEAxMadWJ1bnR1ODA0LWJhc2UubG9jYWxkb21haW4xLjAsBgkqhkiG&#xa;9w0BCQEWH3Jvb3RAdWJ1bnR1ODA0LWJhc2UubG9jYWxkb21haW4wHhcNMTAwMzE3&#xa;MTQwNzQ1WhcNMTAwNDE2MTQwNzQ1WjCB8TELMAkGA1UEBhMCWFgxKjAoBgNVBAgT&#xa;IVRoZXJlIGlzIG5vIHN1Y2ggdGhpbmcgb3V0c2lkZSBVUzETMBEGA1UEBxMKRXZl&#xa;cnl3aGVyZTEOMAwGA1UEChMFT0NPU0ExPDA6BgNVBAsTM09mZmljZSBmb3IgQ29t&#xa;cGxpY2F0aW9uIG9mIE90aGVyd2lzZSBTaW1wbGUgQWZmYWlyczEjMCEGA1UEAxMa&#xa;dWJ1bnR1ODA0LWJhc2UubG9jYWxkb21haW4xLjAsBgkqhkiG9w0BCQEWH3Jvb3RA&#xa;dWJ1bnR1ODA0LWJhc2UubG9jYWxkb21haW4wgZ8wDQYJKoZIhvcNAQEBBQADgY0A&#xa;MIGJAoGBANa0EzYzmpVxexvefIN12nGxPKl//q1kG3fpT66+ytT4y++uu0N5JHP/&#xa;POWeO238yLGs+kxNXptMmVQL16hKULqp3h0f9ORrAqP0a0XNTK+NiWIzj2W7NmGf&#xa;xCxzwU4uoKgUTphwRmG70bkx34yZ7nVreTxAoK6XAJCd3JkNM6S1AgMBAAEwDQYJ&#xa;KoZIhvcNAQEFBQADgYEAkqS0uBRVYyVRSgvDKiLPOvgXagzPZqqnZS9Ibc3jPlyf&#xa;d2zURFQfHoRPjtSN3awtiAkhqNpWLKkFPEloNRl1DNpTI4iIGS10JsEiZe4RaINq&#xa;U0qcJ8ugtOmNKQyyPBhcZ8xTph4w0Komex6uQLkpAWwuvKIZlHwVbo0wOPbKLnU=&#xa;-&#45;&#45;&#45;&#45;END
			 CERTIFICATE-&#45;&#45;&#45;&#45;&#xa;</elem>
			</script><script id="ssl-date" output="2016-03-21T08:38:05+00:00; 
			+1s from scanner time."><elem key="delta">1</elem>
			<elem key="date">2016-03-21T08:38:05+00:00</elem>
			</script></port>
			<port protocol="tcp" portid="53"><state state="open" 
			reason="syn-ack" reason_ttl="64"/><service name="domain" 
			product="ISC BIND" version="9.4.2" method="probed" 
			conf="10"><cpe>cpe:/a:isc:bind:9.4.2</cpe></service><script 
			id="dns-nsid" output="&#xa;  bind.version: 9.4.2"><elem 
			key="bind.version">9.4.2</elem>
			</script></port>
			<port protocol="tcp" portid="80"><state state="open" 
			reason="syn-ack" reason_ttl="64"/><service name="http" 
			product="Apache httpd" version="2.2.8" extrainfo="(Ubuntu) DAV/2" 
			method="probed" 
			conf="10"><cpe>cpe:/a:apache:http_server:2.2.8</cpe></service><script
			 id="http-server-header" output="Apache/2.2.8 (Ubuntu) 
			DAV/2"><elem>Apache/2.2.8 (Ubuntu) DAV/2</elem>
			</script><script id="http-title" output="Metasploitable2 - 
			Linux"><elem key="title">Metasploitable2 - Linux</elem>
			</script></port>
			<port protocol="tcp" portid="111"><state state="open" 
			reason="syn-ack" reason_ttl="64"/><service name="rpcbind" 
			version="2" extrainfo="RPC #100000" method="probed" 
			conf="10"/><script id="rpcinfo" output="&#xa;  program version   
			port/proto  service&#xa;  100000  2            111/tcp  
			rpcbind&#xa;  100000  2            111/udp  rpcbind&#xa;  100003  
			2,3,4       2049/tcp  nfs&#xa;  100003  2,3,4       2049/udp  
			nfs&#xa;  100005  1,2,3      46259/tcp  mountd&#xa;  100005  
			1,2,3      48241/udp  mountd&#xa;  100021  1,3,4      42298/udp  
			nlockmgr&#xa;  100021  1,3,4      47669/tcp  nlockmgr&#xa;  100024  
			1          47986/tcp  status&#xa;  100024  1          51797/udp  
			status&#xa;"><table key="100021">
			<table key="tcp">
			<table key="version">
			<elem>1</elem>
			<elem>3</elem>
			<elem>4</elem>
			</table>
			<elem key="port">47669</elem>
			</table>
			<table key="udp">
			<table key="version">
			<elem>1</elem>
			<elem>3</elem>
			<elem>4</elem>
			</table>
			<elem key="port">42298</elem>
			</table>
			</table>
			<table key="100000">
			<table key="udp">
			<table key="version">
			<elem>2</elem>
			</table>
			<elem key="port">111</elem>
			</table>
			<table key="tcp">
			<table key="version">
			<elem>2</elem>
			</table>
			<elem key="port">111</elem>
			</table>
			</table>
			<table key="100024">
			<table key="tcp">
			<table key="version">
			<elem>1</elem>
			</table>
			<elem key="port">47986</elem>
			</table>
			<table key="udp">
			<table key="version">
			<elem>1</elem>
			</table>
			<elem key="port">51797</elem>
			</table>
			</table>
			<table key="100003">
			<table key="tcp">
			<table key="version">
			<elem>2</elem>
			<elem>3</elem>
			<elem>4</elem>
			</table>
			<elem key="port">2049</elem>
			</table>
			<table key="udp">
			<table key="version">
			<elem>2</elem>
			<elem>3</elem>
			<elem>4</elem>
			</table>
			<elem key="port">2049</elem>
			</table>
			</table>
			<table key="100005">
			<table key="tcp">
			<table key="version">
			<elem>1</elem>
			<elem>2</elem>
			<elem>3</elem>
			</table>
			<elem key="port">46259</elem>
			</table>
			<table key="udp">
			<table key="version">
			<elem>1</elem>
			<elem>2</elem>
			<elem>3</elem>
			</table>
			<elem key="port">48241</elem>
			</table>
			</table>
			</script></port>
			<port protocol="tcp" portid="139"><state state="open" 
			reason="syn-ack" reason_ttl="64"/><service name="netbios-ssn" 
			product="Samba smbd" version="3.X" extrainfo="workgroup: WORKGROUP" 
			method="probed" 
			conf="10"><cpe>cpe:/a:samba:samba:3</cpe></service></port>
			<port protocol="tcp" portid="445"><state state="open" 
			reason="syn-ack" reason_ttl="64"/><service name="netbios-ssn" 
			product="Samba smbd" version="3.X" extrainfo="workgroup: WORKGROUP" 
			method="probed" 
			conf="10"><cpe>cpe:/a:samba:samba:3</cpe></service></port>
			<port protocol="tcp" portid="512"><state state="open" 
			reason="syn-ack" reason_ttl="64"/><service name="exec" 
			product="netkit-rsh rexecd" ostype="Linux" method="probed" 
			conf="10"><cpe>cpe:/a:netkit:netkit</cpe><cpe>cpe:/o:linux:linux_kernel</cpe></service></port>
			<port protocol="tcp" portid="513"><state state="open" 
			reason="syn-ack" reason_ttl="64"/><service name="login" 
			method="table" conf="3"/></port>
			<port protocol="tcp" portid="514"><state state="open" 
			reason="syn-ack" reason_ttl="64"/><service name="shell" 
			product="Netkit rshd" method="probed" 
			conf="10"><cpe>cpe:/a:netkit:netkit_rsh</cpe></service></port>
			<port protocol="tcp" portid="1099"><state state="open" 
			reason="syn-ack" reason_ttl="64"/><service name="java-rmi" 
			product="Java RMI Registry" hostname="localhost" method="probed" 
			conf="10"/></port>
			<port protocol="tcp" portid="1524"><state state="open" 
			reason="syn-ack" reason_ttl="64"/><service name="shell" 
			product="Metasploitable root shell" method="probed" 
			conf="10"/></port>
			<port protocol="tcp" portid="2049"><state state="open" 
			reason="syn-ack" reason_ttl="64"/><service name="nfs" version="2-4" 
			extrainfo="RPC #100003" method="probed" conf="10"/></port>
			<port protocol="tcp" portid="2121"><state state="open" 
			reason="syn-ack" reason_ttl="64"/><service name="ftp" 
			product="ProFTPD" version="1.3.1" ostype="Unix" method="probed" 
			conf="10"><cpe>cpe:/a:proftpd:proftpd:1.3.1</cpe></service></port>
			<port protocol="tcp" portid="3306"><state state="open" 
			reason="syn-ack" reason_ttl="64"/><service name="mysql" 
			product="MySQL" version="5.0.51a-3ubuntu5" method="probed" 
			conf="10"><cpe>cpe:/a:mysql:mysql:5.0.51a-3ubuntu5</cpe></service><script
			 id="mysql-info" output="&#xa;  Protocol: 53&#xa;  Version: 
			.0.51a-3ubuntu5&#xa;  Thread ID: 8&#xa;  Capabilities flags: 
			43564&#xa;  Some Capabilities: Speaks41ProtocolNew, Support41Auth, 
			ConnectWithDatabase, SupportsCompression, SupportsTransactions, 
			SwitchToSSLAfterHandshake, LongColumnFlag&#xa;  Status: 
			Autocommit&#xa;  Salt: *&apos;$}n/&gt;*UMc,O)|V{*\p"><elem 
			key="Protocol">53</elem>
			<elem key="Version">.0.51a-3ubuntu5</elem>
			<elem key="Thread ID">8</elem>
			<elem key="Capabilities flags">43564</elem>
			<table key="Some Capabilities">
			<elem>Speaks41ProtocolNew</elem>
			<elem>Support41Auth</elem>
			<elem>ConnectWithDatabase</elem>
			<elem>SupportsCompression</elem>
			<elem>SupportsTransactions</elem>
			<elem>SwitchToSSLAfterHandshake</elem>
			<elem>LongColumnFlag</elem>
			</table>
			<elem key="Status">Autocommit</elem>
			<elem key="Salt">*&apos;$}n/&gt;*UMc,O)|V{*\p</elem>
			</script></port>
			<port protocol="tcp" portid="3632"><state state="open" 
			reason="syn-ack" reason_ttl="64"/><service name="distccd" 
			product="distccd" version="v1" extrainfo="(GNU) 4.2.4 (Ubuntu 
			4.2.4-1ubuntu4)" method="probed" conf="10"/></port>
			</ports>
			<os><portused state="open" proto="tcp" portid="21"/>
			<portused state="closed" proto="tcp" portid="1"/>
			<portused state="closed" proto="udp" portid="31595"/>
			<osmatch name="Linux 2.6.9 - 2.6.33" accuracy="100" line="53435">
			<osclass type="general purpose" vendor="Linux" osfamily="Linux" 
			osgen="2.6.X" 
			accuracy="100"><cpe>cpe:/o:linux:linux_kernel:2.6</cpe></osclass>
			</osmatch>
			</os>
			<uptime seconds="105" lastboot="Mon Mar 21 04:36:27 2016"/>
			<distance value="1"/>
			<tcpsequence index="205" difficulty="Good luck!" 
			values="1B8A9A35,1B16770E,1B201DD6,1B2C7610,1BF024EA,1B8225C1"/>
			<ipidsequence class="All zeros" values="0,0,0,0,0,0"/>
			<tcptssequence class="100HZ" 
			values="24A4,24AE,24B9,24C3,24CD,24D8"/>
			<hostscript><script id="nbstat" output="NetBIOS name: 
			METASPLOITABLE, NetBIOS user: &lt;unknown&gt;, NetBIOS MAC: 
			&lt;unknown&gt; (unknown)"><table key="names">
			<table>
			<elem key="flags">1024</elem>
			<elem key="suffix">0</elem>
			<elem key="name">METASPLOITABLE</elem>
			</table>
			<table>
			<elem key="flags">1024</elem>
			<elem key="suffix">3</elem>
			<elem key="name">METASPLOITABLE</elem>
			</table>
			<table>
			<elem key="flags">1024</elem>
			<elem key="suffix">32</elem>
			<elem key="name">METASPLOITABLE</elem>
			</table>
			<table>
			<elem key="flags">33792</elem>
			<elem key="suffix">0</elem>
			<elem key="name">WORKGROUP</elem>
			</table>
			<table>
			<elem key="flags">33792</elem>
			<elem key="suffix">30</elem>
			<elem key="name">WORKGROUP</elem>
			</table>
			</table>
			<elem key="server_name">METASPLOITABLE</elem>
			<table key="statistics">
			<elem>00 00 00 00 00 00 00 00 00 00 00 00 00 00 00 00 00</elem>
			<elem>00 00 00 00 00 00 00 00 00 00 00 00 00 00 00 00 00</elem>
			<elem>00 00 00 00 00 00 00 00 00 00 00 00 00 00</elem>
			</table>
			<table key="mac">
			<elem key="manuf">unknown</elem>
			<elem key="address">&lt;unknown&gt;</elem>
			</table>
			<elem key="user">&lt;unknown&gt;</elem>
			</script><script id="smb-os-discovery" output="&#xa;  OS: Unix 
			(Samba 3.0.20-Debian)&#xa;  NetBIOS computer name: &#xa;  
			Workgroup: WORKGROUP&#xa;  System time: 
			2016-03-21T04:38:05-04:00&#xa;"><elem key="os">Unix</elem>
			<elem key="lanmanager">Samba 3.0.20-Debian</elem>
			<elem key="domain">WORKGROUP\x00</elem>
			<elem key="server"></elem>
			<elem key="date">2016-03-21T04:38:05-04:00</elem>
			</script></hostscript><trace>
			<hop ttl="1" ipaddr="192.168.202.2" rtt="1.17"/>
			</trace>
			<times srtt="1168" rttvar="862" to="100000"/>
			</host>
			<runstats><finished time="1458549492" timestr="Mon Mar 21 04:38:12 
			2016" elapsed="38.04" summary="Nmap done at Mon Mar 21 04:38:12 
			2016; 1 IP address (1 host up) scanned in 38.04 seconds" 
			exit="success"/><hosts up="1" down="0" total="1"/>
			</runstats>
			</nmaprun>	
		\end{verbatim}
	\section{Исследование работы утилиты nmap при помощи wireshark}
		При сканировании сети на наличие доступных хостов утилита nmap обращается к DNS серверу для того, чтобы получить список узлов в данной подсети:
		\begin{verbatim}
			7	6.528600840	192.168.202.3	192.168.20.3	DNS	86	Standard 
			query 0x58e6 PTR 2.202.168.192.in-addr.arpa
			8	9.029153317	192.168.202.3	192.168.20.3	DNS	86	Standard 
			query 0x58e7 PTR 2.202.168.192.in-addr.arpa
			9	13.074352796	192.168.202.3	8.8.4.4	DNS	86	Standard query 
			0x58e8 PTR 3.202.168.192.in-addr.arpa
			10	15.576831906	192.168.202.3	8.8.4.4	DNS	86	Standard query 
			0x58e9 PTR 3.202.168.192.in-addr.arpa
		\end{verbatim}
		После получения ответов от DNS сервера, утилита составляет карту доступных узлов в сети.
		
		При простом анализе открытых портов, утилита nmap пытается установить 
		соединение с доступными портами. Если на запрос установления соединения приходит пакет с флагами [SYN, ACK], то порт считается открытым. После этого утилита берет информацию об этом порте из файла nmap-services. Ниже приведен пример сканирования 80 порта (HTTP).
		\begin{verbatim}
			1	0.000000000	192.168.202.3	192.168.202.2	TCP	58	54248 -> 80 
			[SYN] Seq=0 Win=1024 Len=0 MSS=1460
			2	0.002266788	192.168.202.2	192.168.202.3	TCP	60	80 -> 54248 
			[SYN, ACK] Seq=0 Ack=1 Win=5840 Len=0 MSS=1460
			3	0.002281592	192.168.202.3	192.168.202.2	TCP	54	54248 -> 80 
			[RST] Seq=1 Win=0 Len=0
		\end{verbatim}
		Как видно из вывода утилиты wireshark, при сканировании порта, nmap посылает пакет с флагом SYN (флаг запроса на установление соединения), после чего, если в ответ приходит ответ с флагами [SYN, ACK], это означает, что сервер пытается остановить соединение на этом порте. Это означает что порт открыт, и утилита nmap разрывает соединение, посылая пакет с флагом RST (флаг сброса соединения).
		
		Попробуем просканировать закрытый порт. Из предыдущих выводов утилиты nmap видно, что порт 112 закрыт. Попробуем просканировать этот порт, указав его при запуске утилиты при помощи ключа -p:
		\begin{verbatim}
			root@kali:~# nmap 192.168.202.2 -p 112
			
			Starting Nmap 7.01 ( https://nmap.org ) at 2016-03-21 05:54 EDT
			Nmap scan report for 192.168.202.2
			Host is up (0.00041s latency).
			PORT    STATE  SERVICE
			112/tcp closed mcidas
			MAC Address: 08:00:27:3B:18:A4 (Oracle VirtualBox virtual NIC)
			
			Nmap done: 1 IP address (1 host up) scanned in 13.22 seconds
		\end{verbatim}
		Вывод в утилите wireshark:
		\begin{verbatim}
			15	46.978256397	192.168.202.3	192.168.202.2	TCP	58	51420
			-> 112 [SYN] Seq=0 Win=1024 Len=0 MSS=1460
			16	46.978812034	192.168.202.2	192.168.202.3	TCP	60	112
			-> 51420 [RST, ACK] Seq=1 Ack=1 Win=0 Len=0
		\end{verbatim}
		Как видно из вывода утилиты, утилита так же пытается установить соединение по протоколу TCP, посылая пакет с флагом SYN. Так как порт закрыт, в ответ приходит пакет с флагами [RST, ACK]. После этого утилита считает, что порт закрыт, находит информацию об этом порте в файле nmap-services и выводит информацию о нем, однако статус порта будет closed.
\end{document}